\documentclass[]{article}
\usepackage{lmodern}
\usepackage{amssymb,amsmath}
\usepackage{ifxetex,ifluatex}
\usepackage{fixltx2e} % provides \textsubscript
\ifnum 0\ifxetex 1\fi\ifluatex 1\fi=0 % if pdftex
  \usepackage[T1]{fontenc}
  \usepackage[utf8]{inputenc}
\else % if luatex or xelatex
  \ifxetex
    \usepackage{mathspec}
  \else
    \usepackage{fontspec}
  \fi
  \defaultfontfeatures{Ligatures=TeX,Scale=MatchLowercase}
\fi
% use upquote if available, for straight quotes in verbatim environments
\IfFileExists{upquote.sty}{\usepackage{upquote}}{}
% use microtype if available
\IfFileExists{microtype.sty}{%
\usepackage{microtype}
\UseMicrotypeSet[protrusion]{basicmath} % disable protrusion for tt fonts
}{}
\usepackage[margin=1in]{geometry}
\usepackage{hyperref}
\hypersetup{unicode=true,
            pdftitle={The Operating System of Capitalism},
            pdfborder={0 0 0},
            breaklinks=true}
\urlstyle{same}  % don't use monospace font for urls
\usepackage{graphicx,grffile}
\makeatletter
\def\maxwidth{\ifdim\Gin@nat@width>\linewidth\linewidth\else\Gin@nat@width\fi}
\def\maxheight{\ifdim\Gin@nat@height>\textheight\textheight\else\Gin@nat@height\fi}
\makeatother
% Scale images if necessary, so that they will not overflow the page
% margins by default, and it is still possible to overwrite the defaults
% using explicit options in \includegraphics[width, height, ...]{}
\setkeys{Gin}{width=\maxwidth,height=\maxheight,keepaspectratio}
\IfFileExists{parskip.sty}{%
\usepackage{parskip}
}{% else
\setlength{\parindent}{0pt}
\setlength{\parskip}{6pt plus 2pt minus 1pt}
}
\setlength{\emergencystretch}{3em}  % prevent overfull lines
\providecommand{\tightlist}{%
  \setlength{\itemsep}{0pt}\setlength{\parskip}{0pt}}
\setcounter{secnumdepth}{0}
% Redefines (sub)paragraphs to behave more like sections
\ifx\paragraph\undefined\else
\let\oldparagraph\paragraph
\renewcommand{\paragraph}[1]{\oldparagraph{#1}\mbox{}}
\fi
\ifx\subparagraph\undefined\else
\let\oldsubparagraph\subparagraph
\renewcommand{\subparagraph}[1]{\oldsubparagraph{#1}\mbox{}}
\fi

%%% Use protect on footnotes to avoid problems with footnotes in titles
\let\rmarkdownfootnote\footnote%
\def\footnote{\protect\rmarkdownfootnote}

%%% Change title format to be more compact
\usepackage{titling}

% Create subtitle command for use in maketitle
\newcommand{\subtitle}[1]{
  \posttitle{
    \begin{center}\large#1\end{center}
    }
}

\setlength{\droptitle}{-2em}
  \title{The Operating System of Capitalism}
  \pretitle{\vspace{\droptitle}\centering\huge}
  \posttitle{\par}
\subtitle{Revolutionising Tax and Welfare}
  \author{}
  \preauthor{}\postauthor{}
  \predate{\centering\large\emph}
  \postdate{\par}
  \date{\texttt{r\ Sys.Date()}}


\begin{document}
\maketitle

\hypertarget{part-6-tax-in-practice}{%
\section{PART 6: TAX IN PRACTICE}\label{part-6-tax-in-practice}}

\hypertarget{the-existing-tax-system}{%
\subsection{6.1: The Existing Tax
System}\label{the-existing-tax-system}}

Ok, ok. So, we get it's important for taxes to be well designed. We get
that they should be applied strategically, to tax bases that are not
disproportionately impacted by their implementation. They should be
efficient; they should be fair; and most of all they should be part of a
system that provides the greatest amount of benefit possible from the
revenue raised.

We also established that the core purposes of taxation could be reduced
to three streams: revenue-raising, behaviour influencing, and to
redistribute rents and income fairly across the population. That taxes
could be useful in achieving these goals is a theory that economists and
political theorists alike have ascribed to.

But is this the reality in the UK today? With one of the oldest
tax-and-welfare systems in the world, and well-established norms that
facilitate its implementation, it would appear that the UK should be in
a good position to translate these principles into successful practice.

In this chapter we take this claim and analyse it. We look at how the
tax system operates in practice and prove the significance of the claims
we made in the last chapter. After taking a look at each of the
characteristics in turn, we'll be able to evaluate the extent to which
taxes are significant in the UK context.

We'll also take a look at the ways in which taxes fail. As our framework
designates, simplicity and practicality are core components of any
taxation system that we deem to be successful. When large scale evasion
or avoidance occurs, this is a failure of the system. We therefore lay
out a tax avoidance streams. Information is, after all, power. By
clearly outlining the methods by which evasion and avoidance occur, we
hope to create a tool for the citizen and the policy maker alike to hold
individuals accountable.

Taking a systemic perspective of the UK tax and welfare system can be
helpful in determining where the pressure points lie. Tax and welfare
are two sides of the same policy coin. One cannot be fully analysed or
appreciated without the other. Although in our later chapters on
inequality cover the UK welfare system in more detail, here we take a
step back. We look at these policy tools in a holistic manner, gauging
the overall levels of spending and dependency of the benefits system
within this current system.

\hypertarget{revenue-and-reasons}{%
\subsection{6.2 Revenue and reasons}\label{revenue-and-reasons}}

\#\#\#Tax in Practice

The UK tax system consists of approximately 26 different taxes, with a
significant range of tax bases exploited. In total, £594 Billion was
received by the HMRC in 2017/18 (HMRC
\protect\hyperlink{ref-HMRC2019}{2019}). With national GDP in 2018
recorded as £2.033 trillion (ONS \protect\hyperlink{ref-ONS2019}{2019}),
this translates to an overall tax burden of 29.2\%. Of these, income
taxes made up £180 billion, national insurance contributions equated to
£131 billion, and VAT to £125 billion. This demonstrates how
concentrated the UK's tax base is, with over 73\% of total receipts
raised from these three taxes alone.

Even within these concentrated bases, those responsible for the bulk of
income tax are an increasingly diminishing number. According to Full
Fact, the top 1\% of earners were responsible for 28\% of overall income
tax revenues (fact \protect\hyperlink{ref-FullFact2019}{2019}). This
number is of significance when it comes to the economic mentality behind
designing tax policies. Retaining an attractive enough tax environment
for these top earners is a core priority of government tax bodies and is
one of the more important considerations behind `practicality'
considerations in this space. A top tax rate of 100\% may be attractive
to some enthusiastic tweeters, but is, in reality, limited in
application. Ensuring that significant rates of tax are balanced against
the risk of tax evasion and avoidance, and capital flight, is a priority
for policy makers.

Overall, the proportions of total tax receipts received by HMRC in the
UK has remained mostly stable since the 1980s (HMRC
\protect\hyperlink{ref-HMRC2019}{2019}). The key changes of note have
been the transition of the significance in revenues contributed by
indirect taxes, and their replacement by consumption taxes such as VAT.

The wide range of tax bases has, however, remained, making the UK tax
codes one of the most complex in the world. The UK tax code is now 12
times the size of the King James Bible, itself not exactly a nightstand
novella, standing at 22,000 pages on last measurement (Martin
\protect\hyperlink{ref-Martin}{2016}). This is due in part to the Tax
re-write project, which converted previously archaic and disorganised
laws into plain, comprehensible English. Indeed, complexity and length
can be unrelated. The `shorter' US tax system is often thought of as
more complex than that of the UK. However, it remains that the burden of
administration and interpretation is still significant, and there is
substantial political support for simplification of the tax code.

This complexity also has consequences for the HMRC itself, which is the
second-largest government department in terms of staff numbers. The
administrative cost of running this organisation is not insignificant,
running at £3.3 billion in 2016-17 (NAO
\protect\hyperlink{ref-NAO2017}{2017}).

Part of the reason behind the complexity is due to the range of purposes
which the taxes are intended to achieve. Raising revenue for government
spending is clearly the objective of income tax and national insurance
contributions. VAT, to regulate and nudge spending choices. Carbon and
environmental taxes, to change behaviour and regulate pollution. And the
levels of child tax credits, income allowance and housing provision
clearly indicate the goal of redistributing income in the UK.

As it stands, the tax system already makes a significant contribution to
these three goals. Taxes pay for the majority of government spending;
taxes already redistribute income significantly; and taxes on activities
that are harmful to the environment and to health have been levied to
discourage those activities.

However, the tax system is not the only way to achieve these outcomes.
By issuing bonds or imposing user charges, public services could be
funded. Regulation to ensure high quality, readily accessible education,
the strong protection of worker's rights, and changes to property rights
could also help to lower inequality. As for behavioural incentives,
education and cultural change could be highly influential in reducing
unhealthy behaviours, with legal prohibitions applied to those bads
which are seen to be overwhelmingly harmful. Environmental objectives
could also be achieved through government investment and stringent
regulation, rather than market-based instruments.

This is most clear when the tax system fails, and these goals aren't
achieved. Some objectives are too important to leave to the market,
especially when it comes to regulating social and environmental bads.
Moreover, relying on tax revenue can be difficult when the system is as
complex as it currently is. Increased complexity raises the risks of tax
avoidance and evasion for the wealthiest of earners. We'll cover these
later in the chapter.

\#\#\#Welfare in Practice

But what about the flipside? As we've touched on already, the stated
objectives of Beveridge's welfare state were to create a comprehensive
system that employed tax revenue, government policy and moral authority
to tackle the five great evils in society. Today, that is achieved
through the welfare system. Allocating funds to people of diminished
capacity, low income and the elderly creates a safety net for the UK, a
minimum standard of living that people can count on. In theory.

According to the Office for Budgetary Responsibility, the UK public
sector is estimated to have spent £771 billion in 2016-17. Within this,
around £484 billion was allocated to the `welfare state'. Here, we
define the `welfare state' broadly to include `health, education, social
services and housing, as well as social security and tax credits'(Office
for Budget Responsibility \protect\hyperlink{ref-OBR2018}{2018}). The
rest of the spending was on defence, and debt repayments for the
government.

Of the money allocated to the welfare state, some £217 billion was spent
in providing the last two components of this. This social security and
tax credit spending equated to some 28\% of total public spending in
that fiscal year (Office for Budget Responsibility
\protect\hyperlink{ref-OBR2018}{2018}). In other words, this was around
£8000 paid out per household in the UK.

In fact, the UK welfare system is so comprehensive that at some point in
an individual's life, they will most likely receive one or more payments
from the state. These are most commonly in the form of child tax credits
or child benefit, and state pension payments into retirement age. Over
half of families in the UK receive income from one or more welfare
payments in the system, with the majority of these payments (59\%) going
towards pension obligations. Together with personal tax credits, mainly
targeting families with dependents, and housing benefit, these three
areas combine to make around two thirds of total welfare spending. Job
seekers allowance, despite its infamous status, made up only 1\% of the
total spend on social security provisions in total. The figure below
expands on this spending.

{5-1-taxrev}

{5-2-taxage}

As the figure above shows, most of the expenditure on welfare is
directed towards people in retirement, sitting at 40\% of total revenue.
Despite recent policy changes to raise the retirement age, with an aging
population this portion of welfare is set to increase. The rest is
divided amongst those on disability and incapacity benefit, families and
low-income people. These benefits are either on a contributory basis,
i.e.~pension payments that are dependent on National Insurance
contributions), or non-contributory, such as the basic state pension.
These benefits can also be awarded on a means tested basis, i.e.~housing
benefit, or be universally applicable, like universal credit.

As we outlined in the previous chapter, there are three key purposes to
the UK benefits system: to provide a safety net for all individuals at
the bottom of the economic system, to care for specific special need
issues, and to protect individuals into old age with a state pension.
There are obviously flaws in the reality of achieving these goals, most
of which will be expanded on in our inequality chapter.

However, we can summarise by whittling these down to administrative
costs and errors, and issues with inequality and delivery. Food bank use
in the UK has risen 13\% in the 2017-18 period alone, following a 6\%
increase the year previously (The Trussell Trust
\protect\hyperlink{ref-TheTrussellTrust2019}{2019}). This is due in part
to rising costs of living and in administrative issues to do with the
roll out of universal credit, which has left many people behind in their
payment schedule. This will be expanded on in more detail in the chapter
on inequality, but it's important to recognise that there are also
issues in the implementation of these welfare ideals in practice.

\#\#6.2 Bugs in the System

\#\#\#Identifying the issues

There are several problems with the UK tax and benefit system as it
stands today. These are issues that we can categorise into a few groups
as follows: administrative and compliance barriers, tax avoidance and
evasion, and public acceptance of taxes.

A broad complaint about the tax system is to do with its complexity. As
mentioned in our introduction, the over-complexity of the uk tax system
has led to high administration and collection costs, as well as a number
of unwanted consequences to do with progressivity and loopholes built
into the system. The extensive number of exemptions and loopholes which
exist in the system can really be described as government expenditures,
masked as tax rebates. These mainly serve to complicate the system and
should usually be removed.

Moreover, it could be said that despite its length, the system lacks
ambition. As it stands, the tax system doesn't take significant enough
advantage of the opportunities it has to change consumer behaviour. Only
about one sixth of existing tax revenue is raised from taxes which are
aimed at behavioural change. Given the potential for taxation to be used
to create behavioural norms and effect social change, this is a missed
opportunity.

\hypertarget{administrative-issues}{%
\subsubsection{Administrative Issues}\label{administrative-issues}}

The complexity of the current tax system is made more so by the number
of loopholes, exemptions and tax-credits that it offers on a variety of
different criteria. It is difficult for the average citizen to
comprehend the full extent of the tax system as it stands. Many employ
professional services like accountants and lawyers to comprehend its
details. Clearly, citizens cannot hold politicians to account when they
don't understand the system. It is clear then that politically
influential groups may have a great deal of influence in shaping the tax
code.

It is also difficult for the government to administer the policies. The
HMRC is one of the largest government departments, costing over \$3bn
annually to run. The challenges faced by monitoring and enforcing a tax
system which covers over 65 million individuals, with over 26 taxes
applied on direct and indirect tax bases, with what is often a manual
data-management process cannot be understated.

Moreover, due to its complexity, there are several interplays between
different taxes that distort behaviour in an economically inefficient
way. The entire existence of `tax advice' firms shows that there are
economic resources engaged in avoiding tax instead of producing goods
and services, and therefore there is a loss of value to society.

The issues that surround implementation are also significant offshoots
of the system's complexity. Unintended consequences of taxes can arise
from weak communication and poor implementation of new policies.

\#\#\#Public acceptance

The perceptions of tax as an inevitable evil do not aid in its
implementation. Fears of government tax grabs and regressive impacts can
mean that even the most well-designed policies in principle can, in
practice, fail. It is important to consider the communication strategy
as well during the implementation process, and to clearly designate
where the revenues raised from said taxes are allocated to.

A paper published by Demos, a cross-party think tank, suggests a number
of ways that taxes can be designed to make them more politically
acceptable to pay. An overriding comment is that Voters and citizens are
increasingly wary of bureaucratic `black holes' of spending; an issue
that came to light in the MP expenses scandal over the last decade.
Finding that taxpayer's money had been frittered onto, amongst other
things, expensive duck houses was a factor in lowering the legitimacy of
government control over this tax revenue. As the authors themselves
describe, ``Government appears as a black hole into which resources
disappear'' (Mulgan and Murray
\protect\hyperlink{ref-Mulgan1993}{1993}). This paper designates three
mechanisms by which taxes can be made palatable to a voting public.

Firstly, taxes should be `hypothecated'. By this term, the authors mean
that taxes should be clearly linked to the output that they will be
directed towards. We can see this in the case of carbon-tax theory,
which a number of studies have examined to determine the most
politically viable avenues to allocate these new revenues to. A recent
study by the Oxford Martin School found that how the design of carbon
pricing reforms incorporated this issue was directly linked to their
longer-term success (Our World in Data
\protect\hyperlink{ref-OurWorldinData2018}{2018}). An overarching
finding was the visibility of revenue use was a key factor, which could
materialise in the form of green infrastructure investment, tax rebates
and consumer subsidies, or even direct transfers to households. The
latter methods have the benefit of tackling the potentially regressive
impacts of carbon taxes which, as a consumption-based tax, can target
poorer households more significantly than richer consumers.

By `regressive' this means that the burden of the tax falls more
considerably on the poor, than on the rich. Even though in absolute
terms, the latter may be paying more, as a percentage of disposable
income. Taxes like VAT, National Insurance, Vehicle Excise Duty and
council tax fall under this category. British Columbia is an example of
success in this respect, having directly rebated households with the
proceeds from their nascent carbon taxation scheme. Public acceptance
rates of this scheme have thus been maintained. Compare this to the
poorly communicated fuel levy placed in France towards the end of 2018,
and the resultant `Gilet Jaune' protests, and the importance of clear
communication about the purposes to which a policy is put become clear.

{5-3-taxtrust}

The figure above correlates the height, or price, of carbon prices in
governments with public trust in government and politicians. Carbon tax
rates are positively correlated with trust in politicians, and
negatively correlated with perceptions of corruption (Our World in Data
\protect\hyperlink{ref-OurWorldinData2018}{2018}).

The second recommendation from Demos' paper was that voters `should have
more influence on spending choices\ldots{} spending should be decided by
referendums' (Mulgan and Murray
\protect\hyperlink{ref-Mulgan1993}{1993}). This is potentially a way to
bypass allegations of corruption and low public trust. However, the
administrative cost of such a system is not to be dismissed.

Thirdly, local spending should be tied to local taxes, so that high
levels of visibility can be maintained for local voters. Councils should
be allowed to determine local spending to an extent that is agreed on by
their electorate. This would allow the level of autonomy and connection
to the application taxes needed to maintain elevated levels of
commitment. This feeling of buy-in is not only important on the lower
income levels of UK society, as we shall explore in the next section on
tax avoidance. Citizen buy-in is an important factor for ensuring
sustainability within the system.

\#\#\#Tax avoidance strategies

``Tax'', as said Theresa May in 2017, ``is the price we pay for living
in a civilised society (Houlder
\protect\hyperlink{ref-Houlder2017}{2017}). Speaking in the wake of the
tax avoidance scandal that implicated a number of well-known
celebrities, including Jimmy Carr and Bono, the Prime Minister spoke for
a cross-party majority when she attacked those who evaded and avoided
tax. However, with the large number of reliefs, loopholes and rebates
that exist in the over-complex system

As it stands, the HMRC relies on self-reported income, and could
therefore be termed an honesty scheme. There are regular check ups, used
to incentivise people into truthfully reporting their income. However,
the sheer scale of the enterprise, combined with the ability for
wealthier individuals to use complex taxation schemes, means that the
administrative body is often playing catch up with enforcing these
regulations. The system's loopholes encourage avoidance and has not,
until recently, strongly punished evasion. Richer individuals and
companies are incentivised to pay advisers to avoid tax legally (or even
illegally). Large companies can take advantages of variations in tax
treatments between different countries (`tax arbitrages'). It must be
said then that changes in a single country's tax policies cannot prevent
this alone. However, simplification of the tax system in the UK could
help at least in making this clearer for officials and the public,
aiding in identifying those entities which engage in tax arbitrage
practices.

The HMRC has started to tackle tax avoidance more strongly in recent
years, with `tough new legislation and demands that they pay disputed
tax up front' (Houlder \protect\hyperlink{ref-Houlder2017}{2017}).
Recent prosecutions and stricter controls have closed off many of the
avenues that were once common place for large multinationals and
wealthier individuals. The world of tax evasion and avoidance is
smaller, and mostly plays closer to the letter of the law -- layering
the available relief schemes available from the government. It still
remains, however, that there are a significant number of tax-saving
ploys that allow and encourage high-net-worth individuals to mitigate
their tax liabilities.

Knowledge is power. Especially when it comes to tax evasion. It
therefore seems relevant to outline the most common methods of tax
evasion and avoidance that high-net worth individuals engage in. The
HMRC itself regularly publishes a breakdown of the tax evasion and
avoidance strategies which it is aware of (HMRC
\protect\hyperlink{ref-HMRC2018b}{2018}).

\begin{enumerate}
\def\labelenumi{\arabic{enumi})}
\item
  Tax Havens
\item
  Shell companies
\item
  Equity swaps
\item
  Avoiding capital gains tax
\item
  Evading estate tax/inheritance tax
\item
  Shell trust funds
\item
  Incorporating
\item
  Payments in kind
\item
  Life insurance borrowing
\item
  Real estate borrowing
\end{enumerate}

\#\#6.3 Conclusions

It is clear then that from this chapter, the tax and benefit system does
not operate in practice the way it is laid out to in principle. The best
laid plans of mice and men can fail to fully enforce behavioural taxes
on a wide and mobile tax base, as the old saying goes.

It's important then to draw out some recommendations for how policy
makers should be adapting the tax system to combat these implementation
issues. Communication, clarity and enforcement are all aspects of this
transition.

What is taxed should be concrete and objective. It should not depend on
the honesty of those declaring the tax and it should not penalize the
honest relative to the dishonest. The tax system should not contain `tax
arbitrages' and should not allow for dishonesty. This would help to
combat both the issues of public acceptance and administrative
complexity. The costs and time-commitment of administering such a system
should never be excessive. Streamlining the tax code to simplify the
number of tax rates, directly tie benefits to the purposes of taxes that
are applied, and a limited number of exemptions should be available.
Above all, the tax and benefit system should be capable of being easily
understood by the public.

In terms of public acceptance, the purposes and targets of taxes should
be well communicated and clear. As outlined in the previous chapter,
following clear principles when taxes are created so that they are
practical, beneficial and fair is essential when ensuring that they will
be successfully launched. Tax should either fall on things that don't go
away when you tax them (e.g.~land) or things that you want to go away
when you tax them (e.g.~fossil fuel use sugar consumption).

It is important that the tax and benefit system works. As we've shown,
income tax and consumption tax revenue are the life blood of public
services like the NHS, the educational system and for social security.
There is buy in on the welfare side. Most individuals in this country
will benefit from a public payment of some kind, from healthcare to
child benefit to the state pension, in their lifetimes. Creating the
same level of connection with the taxation system is an essential
component of the system's sustainability. These key mechanisms by which
the UK raises revenue, redistributes income, and nudges people towards
the right choices have been neglected and left to rust, out of date with
the times we live in. There is no shortage of popular support for
reform. Indeed, consensus seems to be on the side of the need for
improving and simplifying the current tax and welfare system.

As we will explore in the coming chapters, in more detail, things are
changing, but not significantly enough. Proposals put in place that aim
to move towards a universal credit system have ignored the need for
changes to be practical, beneficial and fair. HMRC's tightening up of
tax evasion has yet to fully solve the problem. And destructive
behaviours have yet to fully be targeted through the use of taxation.
There is still a significant way to go with realising our principles in
practice.

\hypertarget{refs}{}
\leavevmode\hypertarget{ref-FullFact2019}{}%
fact, Full. 2019.
\url{https://fullfact.org/economy/do-top-1-earners-pay-28-tax-burden/}.

\leavevmode\hypertarget{ref-HMRC2018b}{}%
HMRC. 2018. ``Tax avoidance schemes currently in the spotlight (numbers
1 to 19).''
\url{https://www.gov.uk/government/collections/tax-avoidance-schemes-currently-in-the-spotlight}.

\leavevmode\hypertarget{ref-HMRC2019}{}%
---------. 2019. ``HMRC tax receipts and National Insurance
contributions for the UK.''
\url{https://www.gov.uk/government/statistics/hmrc-tax-and-nics-receipts-for-the-uk}.

\leavevmode\hypertarget{ref-Houlder2017}{}%
Houlder, Vanessa. 2017. ``Fifteen ways to reduce your tax bill.''
\url{https://www.ft.com/content/bb8a6fa2-e651-11e6-893c-082c54a7f539}.

\leavevmode\hypertarget{ref-Martin}{}%
Martin, David. 2016. ``A New, Simple, Revenue Neutral Tax Code for
Business.'' \emph{Centre for Policy Studies}.
\url{http://www.cps.org.uk/files/reports/original/160304113651-ANewSimpleRevenueNeutralTaxCodeforBusiness.pdf}.

\leavevmode\hypertarget{ref-Mulgan1993}{}%
Mulgan, Geoff, and Robin Murray. 1993. ``How to make paying taxes more
palatable.''
\url{https://www.independent.co.uk/voices/leading-article-how-to-make-paying-taxes-more-palatable-1491568.html}.

\leavevmode\hypertarget{ref-NAO2017}{}%
NAO. 2017. ``HM Revenue \& Customs 2016-17 Accounts: Report by the
Comptroller and Auditor General.''
\url{https://www.nao.org.uk/wp-content/uploads/2017/07/HM-Revenue-Customs-2016-17-Accounts-Report-by-the-Comptroller-and-Auditor-General.pdf}.

\leavevmode\hypertarget{ref-OBR2018}{}%
Office for Budget Responsibility. 2018. ``An OBR Guide to Welfare
Spending.''
\url{http://obr.uk/forecasts-in-depth/brief-guides-and-explainers/an-obr-guide-to-welfare-spending/}.

\leavevmode\hypertarget{ref-ONS2019}{}%
ONS. 2019. ``Gross Domestic Product: chained volume measures: Seasonally
adjusted £m - Office for National Statistics.''
\url{https://www.ons.gov.uk/economy/grossdomesticproductgdp/timeseries/abmi/qna}.

\leavevmode\hypertarget{ref-OurWorldinData2018}{}%
Our World in Data. 2018. ``Why is carbon pricing in some countries more
successful than in others?''
\url{https://ourworldindata.org/carbon-pricing-popular}.

\leavevmode\hypertarget{ref-TheTrussellTrust2019}{}%
The Trussell Trust. 2019. ``End of Year Stats - Food Bank Use.''
\url{https://www.trusselltrust.org/news-and-blog/latest-stats/end-year-stats/}.


\end{document}
